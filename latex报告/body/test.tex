\chapter{测试与开发过程}
\section{传播延时方面}
\begin{itemize}
    \item 最初方案:对float数组元素逐个传输,传完一次数据约耗时2s。
    \item 原因是每传输一次数据都会调用一次Wi-Fi模块,经过测试,调用Wi-Fi模块是占用整个传输时间最大比例的,所以后续改进都从减少调用Wi-Fi模块次数的角度考虑。
    \item 改进方案:将float数组拼接为字符串,传完一次数据约耗时0.6s。
    \item 该方案确实减少了Wi-Fi模块调用时间从而大大减少了传输时间。但是在对数据进行类型转换的时候,由float型(4字节)转换为对应的string型(8字节)时,数据的字节数发生了增长,后续改进都针对数据字节规模进行了考虑。考虑尝试使用UDP进行传输,但是UDP丢包问题可能使得服务器在单位时间内接收的数据量更少,得不偿失。后又考虑把几次数据合为一个字符串进行传输,但是在网络协议中包的大小有着明确的限制,不论合成几次数据都会分为多个包进行发送,并且合成数据时也浪费了大量的时间,经过考虑后,放弃了UDP的传输方式。
    \item 最终方案:在ESP32上对float做乘100运算后类型转换为int,而后拼为字符串进行传输,传完一组数据约耗时0.3s。
    \item 这个方法降低了转换后字符串的长度,并且后续处理数据也更为简单方便。
    \end{itemize}
\section{Wi-Fi连接方式}
\begin{itemize}
    \item 最初方案:使用ESP32开设热点即AP模式,接收设备连接进行数据接收。
    \item 但是由于ESP32本身没有网络环境,外部设备连接后会使得设备无法访问外部网络,最后放弃了这个方案。
    \item 改进方案:ESP32调用自身Wi-Fi模块,利用事先输入好的Wi-Fi名称和密码连接对应网络,外部设备连接同一个Wi-Fi网络,在局域网内进行数据传输。
    \item 适用范围提升了许多,但是每次更换网络环境,需要重新烧写ESP32,十分繁琐。
    \item 第二次改进:ESP32的AP模式可以使其作为服务器向外部连接的提供一个html网页来输入想要连接的网络名称和密码。
    \item 虽然适配了复杂多变的网络环境,但是每次断电,启动都需要输入一次环境配置,也不够人性化。
    \item 最终方案:保存上次连接的Wi-Fi信息,启动后先尝试连接上次连接的Wi-Fi,不成功再开启AP模式,用外部设备连接,输入新的网络配置。
    \end{itemize}
\section{数据接收服务}
\begin{itemize} 
    \item 最初方案:使用Java编写的Service服务器,与ESP32建立Socket连接。
    \item 每次ESP32重新启动时,会递交新的Socket,但是之前的服务器仍然在等待上一次建立连接的Socket的输入流,只能重启服务器才能再次进行连接。
    \item 最终方案:建立两个Service对象交替使用,在一个服务器建立连接后,另一个服务器依然在监听Socket请求,当收到新的请求后,结束上一个服务器的线程,并且再次开启一个新的线程用于监听。
\end{itemize}   
\section{插值算法的选择}
\begin{itemize}
    \item 最初方案:简单的取平均值算法。
    \item 速度很快,计算压力较小,实现了图像分辨率的增加,但是计算的参照物少,精度不高,生成的图像比较模糊,细节缺失较多。
    \item 最终方案:二元和三元插值算法。
    \item 相较于之前的算法得出的图像锯齿感有所减少。尽管三元插值算法在图像细节上可能有较好的呈现,但是因为其运算过程复杂,可能导致数据从传输到最终显示在前端页面上有着较大的延迟,所以最后选择了耗时较少的二元插值算法,让图像质量和显示延时取得较好的平衡。
    \end{itemize}
\section{温度区间色彩}
\begin{itemize}
    \item 最初方案:简单的让温度在色彩条带上均匀分布。
    \item 由于温度跨度较大,而实际测温时的温度区间差值较小,使得人和环境的温度都落在色带上较为接近的位置,最终渲染出来的图像颜色较为接近,区分度不高。原本设想温度低于某一个阈值时就将其颜色置为纯色,使得背景颜色不影响主要测温物体颜色的呈现。但是由于测温物体边缘的温度有时与环境相差无几,细节缺失较多,最后呈现出来的图像也并没有达到预想的效果。
    \item 最终方案:根据实际情况,将人体体温附近的温度在色彩区间上占比增加,调整了其他温度区间的色彩区间占比,显著提升了成像的质量。
    \end{itemize}
\section{图片的产生和输出}
\begin{itemize}
    \item 最初方案:将经过处理后的数据封装入image对象中,在本地生成图片,再将本地生成的图片转为Base64编码。
    \item 但是图片刷新速率快,每次运行产生大量图片,给调试者带来了不必要的麻烦。并且由于生产图片需要根据路径存放,也不易于服务器的移植。
    \item 最终方案:将生成图片封装的image对象直接转换为字节流的形式,再对字节流进行Base64编码的转换。
    \item 最终使得图片流式传输,不占用本地资源,同时也节省了图片生成及读取的时间,减少了前端页面的延迟。
    \end{itemize}
\section{服务器的部署}
    \begin{itemize}
    \item 最初方案:服务器部署于本机。
    \item 必须通过局域网才能访问,硬件部分每次重新启动服务器也需要重新启动。即使通过端口映射到学校网络实现连接学校网络即可访问,也并不是十分稳定。
    \item 最终方案:将服务器部署在阿里云ESC实例上,使其可以通过公网ip访问。
    \item 只要有网络环境和浏览器的设备即可访问。
    \end{itemize}
\section{前端页面的改进}
\begin{itemize}
    \item 最初方案:在进行html编写时采用固定像素的方式布局页面。
    \item 在常见PC端可以正常展现画面,但是在其他移动设备上页面效果非常僵化,页面布局混乱。
    \item 最终方案:进行整个页面的重新设计,设定html网页布局,分配好容器,元素大小使用容器相对值进行设定。
    \item 使得Web适配了PC端、移动端,在不同的设备上使用也能获得较为良好的使用体验,真正实现了多端访问的需求。
    \end{itemize}
