\chapter{测试与开发过程}
\section{传播延时方面}
\begin{itemize}
    \item 最初方案:对float数组逐个传输,传完一次数据约耗时2s
    \item 改进方案:将float数组拼接为字符串,传完一次数据约耗时0.6s
    \item 尝试:使用udp进行传输,受包的大小限制,且不可靠传输导致顺序的错乱,放弃
    \item 最终方案:在ESP32上对float做乘100后类型转换为int,而后拼为字符串进行传输,降低了字符串的长度,传完一组数据约耗时0.3s
    \end{itemize}
\section{WiFi传输方案的选择}
\begin{itemize}
    \item 最初方案:使用ESP32开设热点,电脑连接进行数据发送。缺点:电脑不能上网
    \item 改进方案:使ESP32和电脑连接同一个局域网。缺点:每次更换网络环境,需要重新烧写。
    \item 第二次改进:先让ESP32开设热点,作为服务器提供一个html网页来输入网络名称和密码。缺点:每次启动都需要输入一次。
    \item 最终方案:保存上次连接的WiFi,启动后先尝试连接上次连接的WiFi,不成功再开启热点连接新的网络。
    \end{itemize}
\section{插值算法的选择}
\begin{itemize}
    \item 最初方案:简单的取平均值算法,精度不高,生成的图像比较模糊,细节缺失较多。
    \item 最终方案:实现了二元和三元插值算法,由于两种算法得到的图像差距较小,且二次插值算法在时间复杂度上有较大的优势,因此选择了二次插值算法。
    \end{itemize}
\section{温度区间色彩}
\begin{itemize}
    \item 最初方案:简单的让温度在色彩条带上均匀分布,由于温度跨度较大,而实际测温时的温度范围比较小,使得人和环境难以区分。
    \item 尝试:将温度低于一个值的颜色置为黑/白色,细节缺失较多,观感也不佳。
    \item 最终方案:根据实际情况,将人体体温附近的温度在色彩区间上占比增加,调整了其他温度区间的色彩区间占比,显著提升了成像的质量。
    \end{itemize}
\section{图片的产生和输出}
\begin{itemize}
    \item 最初方案:先在本地生成图片,再对本地生成的图片转为Base64编码。由于刷新速率快,每次运行产生大量图片。
    \item 最终方案:将生成的图片赋给一个image对象,直接对其进行Base64编码的转换。
    \end{itemize}
\section{服务器的部署}
    \begin{itemize}
    \item 最初方案:服务器部署于本机。缺点:必须通过局域网才能访问,硬件部分每次重新启动服务器也需要重新启动。
    \item 改进方案:将服务器部署在阿里云上,使得可以通过公网ip访问。
    \item 最终方案:设立两个服务器轮流运行,每次重启时连接另一个服务器并关闭前一个服务器进程,使得服务器可以一直运行,随用随连。
    \end{itemize}
\section{前端页面的改进}
\begin{itemize}
    \item 使得Web适配了电脑、手机、平板,真正实现了多端访问的需求。
    \end{itemize}
