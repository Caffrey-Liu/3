\chapter{引言}
\section{背景}
2019年12月以来,湖北省武汉市部分医院陆续发现了多例有华南海鲜市场暴露史的不明原因肺炎病例,现已证实为2019新型冠状病毒感染引起的急性呼吸道传染病。新型冠状病毒感染,初期症状可能和感冒差不多,可能有咳嗽,咳痰,发热,胸闷呼吸道症状。可能有腹痛,腹泻消化道症状。可能有乏力,四肢酸痛全身症状。可能有眼红,流泪眼部症状。因此,为了更好的防控防治,无论是个人及家庭,或是社区、车站、医院等地方都急需测温来筛查发热病人,因此测温仪需求量大增。

红外测温仪由光学系统、光电探测器、信号放大器及信号处理、显示输出等部分组成。光学系统汇聚其视场内的目标红外辐射能量,视场的大小由测温仪的光学零件及其位置确定。红外能量聚焦在光电探测器上并转变为相应的电信号。该信号经过放大器和信号处理电路,并按照仪器内疗的算法和目标发射率校正后转变为被测目标的温度值。与传统接触式温度计相比,红外线测温仪有着响应时间快、非接触、准确测量、使用安全及使用寿命长等优点,结合成熟的软件技术,更能使其发挥传统测温仪难以实现的优越性能。

目前应用较为广泛的主要是单点式的非接触红外测温系统,虽然能满足一般需求,但由于没有一个对人体温度的全面感知,存在漏检问题。红外热像仪是红外测温仪的升级版本,它可以将人体表面的温度发布用彩色图像的形式输出到显示器或屏幕上,让我们可以直接“看见”温度,不同的色彩代表着不同的温度,温度高低一目了然。红外成像仪适用于人流量很大的公共场合的人群排查,在保证更高精度的同时,也提高了测温的效率。本课题提出使用MLX90640红外热像传感器的热像体温测量方案。
\section{开发目的}
本项目主要目标为:以MLX90640传感器和ESP32完成红外数据的采集,在Web端实时显示热成像与关键数据,同时可以对出入人员的信息进行登记与查询。适合放置于商场入口、公司大门等人流量多,对测温需求大的地点。
